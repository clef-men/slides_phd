\section{Conclusion}

\begin{frame}{Takeaway}
\Large
\textbf{\Iris-based verification frameworks can scale to real-life programming languages and large pieces of software.}
\vfill
\begin{itemize}
  \item Practical verification framework.
  \item Formalization of a realistic \OCaml subset.
  \item Verification of realistic concurrent data structures.
  \item Extensions of \Iris proof techniques.
  \item \OCaml language improvements.
\end{itemize}
\begin{overbox}<2>[width=.95]
  \textbf{Not mentioned in this presentation:}
  \begin{itemize}
    \item Memory safety.
    \item ``Tail Modulo Cons'' program transformation.
    \item Ongoing work on the verification of (parts of) the \OCaml GC.
  \end{itemize}
\end{overbox}
\end{frame}

\begin{frame}{Future work}
\Large
\begin{itemize}
  \setlength\itemsep{1em}
  \item
    \textbf{Language features}
    \begin{itemize}
      \item Exceptions
      \item Algebraic effects
      \item Modules \& functors
    \end{itemize}
  \item
    \textbf{Coupling with semi-automated verification}
  \item
    \textbf{Relaxed memory}
\end{itemize}
\end{frame}

% transfer resources *and* memory views
\begin{frame}{Relaxed memory}
\Large
{
  \bfseries
  Methodology by Glen Mével \etal:
}
\bigskip
\begin{enumerate}
  \setlength\itemsep{1em}
  \item
    Start with the invariant under sequential consistency;
  \item
    Identify how information flows between domains, \\
    \ie where the \textcolor{color2}{synchronization} points are;
  \item
    Refine the invariant with \textcolor{color2}{memory views} accordingly.
\end{enumerate}
\end{frame}
